\documentclass[a4paper]{article}

\usepackage[utf8]{inputenc}
\usepackage[T1]{fontenc}
\usepackage[russian]{babel}

\usepackage{aleph-comandos}
\usepackage{aleph-moodle}
\usepackage{amsmath}
\usepackage{amssymb} 
\usepackage{marvosym}
\usepackage{setspace}

\usepackage{hyperref}
\hypersetup{
    colorlinks=true,
    linkcolor=blue,
    filecolor=magenta,      
    urlcolor=cyan,
}
\urlstyle{same}

% Se recomienda leer la documentación de estos
% paquetes en https://www.alephsub0.org/recursos/

% -- Paquetes adicionales

\title{Линейная регрессия}
\author{Ельцов Данил, Михаил Михайлов}
\renewcommand*\contentsname{Содержание}

\usepackage[margin=1cm,]{geometry}
\setlength{\parindent}{3em}
\setlength{\parskip}{1em}
\renewcommand{\baselinestretch}{1.2}

\begin{document}
\maketitle

\tableofcontents

\Large \textbf{Резюме}
\par По датасетам с Kaggle и официальной статистики по COVID-19 в мире была построена регрессионная модель, способная по демографическим данным страны предсказать кривую развития
пандемии в отдельно взятой стране.\\
На вход модель принимает числовые характеристики конкретной страны и результатом работы программы являются коэффициенты логистической прямой
\clearpage

\section{Постановка задачи}
Предсказать динамику роста новой коронавирусной инфекции в конкретной стране, основываясь на ее географических и демографических особенностях. 
\section{Используемые данные}
Был взят \href{https://www.kaggle.com/daniboy370/world-data-by-country-2020}{датасет}, содержащий числовые характеристики самых больших стран с Kaggle. Представляет из себя несколько таблиц одинакогового формата: \textbf{id}, \textbf{country}, \textbf{country\_code}, \textbf{feature}\\
Также был взят \href{https://www.ecdc.europa.eu/en/publications-data/download-todays-data-geographic-distribution-covid-19-cases-worldwide}{датасет}, содержащий официальную статистику по развитию коронавируса в разных странах.
\section{Описание решения}
\subsection{Идея решения}
В результате анализа темпов развития COVID-19 в различных странах было сделано предположение, что развитие коронавируса в целом происходит согласно логистической кривой. В следствие этого было решено построить регрессионную модель для предсказания её параметров.
\subsection{Подготовка данных}
Для начала необходимо было объединить все характеристики стран в один CSV-файл, что легко было сделано с помощью их трех-буквенного кода. Затем мы к каждой стране из этой таблицы сопоставили посчитанные для неё коэффициенты логистической регрессии. В результате получился файл \textbf{clear\_data.csv}, имеющий следующую структуру:
\[(ISO-code|rfactor|Median-age|Sex-ratio|Urbanization -rate)\]
\subsection{Вычисление коэффициентов}
Дифференциальное уравнение процесса выглядит следующим образом
\[\frac{dP}{dt} = rP(1-\frac{P}{K})\]
где 
\begin{itemize}
    \item P - количество зараженных
    \item r - коэффициент роста
    \item K - поддерживающая емкость среды
\end{itemize} 
Поддерживающую емкость среды, как критическое значение заболевших было решено взять значение, после которого рост устремится к нулю, у нас оно предполагается равным \(0.75 * P_{max}\), где \(P_{max}\) - число жителей в данной стране.
\subsubsection{Теоретический расчет параметров}
После применения некоторых алгебраических операций и усреднений мы выводим формулу для расчёта коэффициента роста r
\[r = 2 * \frac{1}{n}\sum_{i=1}^{n} \frac{dP_i}{P_i}\]
\subsubsection{Предсказание целевой переменной}
Для предсказания коэффициента $r$ логистической кривой конкретной страны будут использованы её следующие демографические признаки:
\begin{itemize}
    \item sex\_ratio
    \item median\_age
    \item urbanization\_rate
\end{itemize}
\subsection{Построение регрессионной модели}
    После предпосчета параметров регрессии для каждой страны мы запустили обучение модели \href{https://scikit-learn.org/stable/modules/generated/sklearn.linear_model.LinearRegression.html}{LinearRegression} из популярной библиотеки для машинного обучения - \textbf{sklearn}, которая подобрала коэффициенты регрессионной кривой, минимизирующий средний квадрат ошибки. Мы сохранили веса модели линейной регрессии для осуществления дальнейших предсказаний с ее помощью, используя библиотеку \href{https://joblib.readthedocs.io/en/latest/}{joblib}.

\section{Результаты}
\begin{itemize}
    \item Подборка датасета - Данил Ельцов, Михаил Михайлов
    \item Анализ данных - Данил Ельцов, Михаил Михайлов
    \item Построение модели - Михаил Михайлов
    \item Отчёт - Данил Ельцов
    \item\href{https://github.com/Desiment/ml-study/blob/main/linear_regression}{Code}
\end{itemize}

\end{document} 