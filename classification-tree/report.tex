\documentclass{article}
\usepackage[T2A]{fontenc}
\usepackage[russian]{babel}
\usepackage[utf8]{inputenc}

%%%%%%%%%%%%%%%%%%%%%%%%%%%% ДОП.СИМВОЛЫ  %%%%%%%%%%%%%%%%%%%%%%%%%%%%%%%%
\usepackage{amsmath}
\usepackage{amssymb}
\usepackage{latexsym}
\usepackage{amsfonts}
\usepackage{extarrows}
\usepackage{braket}
\usepackage{MnSymbol}
\usepackage{mathtools}
\usepackage{commath}

\DeclarePairedDelimiter{\ceil}{\lceil}{\rceil}
\DeclarePairedDelimiter{\floor}{\lfloor}{\rfloor}
%%%%%%%%%%%%%%%%%%%%%%%%%%%%%%%%%%%%%%%%%%%%%%%%%%%%%%%%%%%%%%%%%%%%%%%%

%%%%%%%%%%%%%%%%%%%%%%%%%%%%%  ГРАФИКА  %%%%%%%%%%%%%%%%%%%%%%%%%%%%%%%%
%Цвета:
\usepackage{color} 
\usepackage{xcolor}

%Картиночки:
\usepackage{graphicx}
\graphicspath{{pictures/}}
\DeclareGraphicsExtensions{.pdf,.png,.jpg}

%Встроенная графика 
\usepackage{tikz}
\usetikzlibrary{
    shapes.symbols,
    shapes.geometric,
    shadows,arrows.meta,
    graphs
}
\usepackage{flowchart}
%%%%%%%%%%%%%%%%%%%%%%%%%%%%%%%%%%%%%%%%%%%%%%%%%%%%%%%%%%%%%%%%%%%%%%%%

%%%%%%%%%%%%%%%%%%%%%%%%%%%%%% ВЕРСТКА 1 %%%%%%%%%%%%%%%%%%%%%%%%%%%%%%%%%
\usepackage[toc,page]{appendix}
\usepackage{hyperref}
\hypersetup{
    unicode=true,
    colorlinks=true,
    linktoc=all,  
    linkcolor=blue,
}
\usepackage{hhline}
\usepackage{subcaption}
\usepackage{float}
\usepackage{enumitem}
%%%%%%%%%%%%%%%%%%%%%%%%%%%%%%%%%%%%%%%%%%%%%%%%%%%%%%%%%%%%%%%%%%%%%%%%

%%%%%%%%%%%%%%%%%%%%%%%%%%%%%% ВЕРСТКА 2 %%%%%%%%%%%%%%%%%%%%%%%%%%%%%%%%%
% Шрифты - настройки по умолчанию.
\renewcommand{\rmdefault}{cmr}
\renewcommand{\sfdefault}{cmss}
\renewcommand{\ttdefault}{cmtt}

%Формат секции
\makeatletter
\renewcommand{\@seccntformat}[1]{}
\makeatother


%Пробел
\setlength{\parindent}{0pt}
\setlength{\parskip}{3pt}

%Размеры страницы (не забыть подогнать под принтер)
\usepackage[left=2cm,right=2cm,bottom=2cm]{geometry}

%Списки:
\setlist{topsep=1pt, itemsep=0em}
%%%%%%%%%%%%%%%%%%%%%%%%%%%%%%%%%%%%%%%%%%%%%%%%%%%%%%%%%%%%%%%%%%%%%%%%%%%%%%%%%%%%%%%%%%%%

\title{Отчет по домашней работе от 03.11.20}
\author{Михаил Михайлов, Ельцов Даниил}
\date{10 ноября 2020 г.}
\begin{document}
\maketitle
\tableofcontents

\section*{Резюме}
Был построен классификатор, который по количеству дипломов от школы по каждому предмету определяет ее профиль. Источник данных --- датасет победителей и призеров олимпиад московских школ (\href{https://www.kaggle.com/romazepa/moscow-schools-winners-of-educational-olympiads}{источник}). 

На вход программы подается название школы, после чего строится вектор данных, содержащий информацию о количестве дипломов (диплом призера и победителя считаются равноценными), который прогоняется через классификатор. 

\newpage
\section{Постановка задачи}
Построить классификатор который определит к какому из профилей принадлежит школа:
\begin{itemize}
    \item физико-математический,
    \item естественно-научный,
    \item обществоведчский,
    \item языковый,
    \item гуманитарный,
    \item многопрофильный/без профиля/не определенно.
\end{itemize}
\section{Используемый датасет}
Датасет взят с \url{kaggle.com} --- \href{https://www.kaggle.com/romazepa/moscow-schools-winners-of-educational-olympiads}{источник}.

Состоит из одной csv таблицы, содержащей около 70 тысяч строк, формата: \\
\small{\texttt{Полное название школы, Краткое название школы, Тип олимпиады (ММО/Всеросс), Этап, Класс, Предмет, Год, ID}}

\section{Описание решения}
\subsection{Разметка предметов по профилям}
\begin{itemize}
    \item \textit{Категория физико-математических профилей}: астрономия, физика, информатика, математика, информатика и информационно-коммуникационные технологии (ИКТ).
    \item \textit{Категория естественно-научных профилей}: биология, географияя, химия, экология.
    \item \textit{Категория обществоведчских профилей}: изобразительное искусство, история, мировая художественная культура (МХК), обществознание, право, экономика, искусство (МХК), бюджетная грамотность.
    \item \textit{Категория языковых профилей}: французский язык, итальянский язык, китайский язык, английский язык, испанский язык, немецкий язык, латынь.
    \item \textit{Категория гуманитарных профилей}: филология, русский язык, литература, лингвистика.
    \item \textit{Категория других предметов}: технология, основы безопасности жизнедеятельности, физическая культура, робототехника, информационные технологии в профессиональной деятельности.
\end{itemize}

\subsection{Обработка данных}
Строится таблица: \\
\small{\texttt{Полное имя школы --- вектор: (Количество дипломов категории 1, \dots, Количество дипломов категории 6)}}


\subsection{Построение классификатора}
Для классификатора зафиксированы константы:
\begin{itemize}
    \item \texttt{THRESHOLD} --- порог числа дипломов, необходимого для того чтобы можно было определить профиль. 
    \item \texttt{MAIN\_SUBJECT\_THRESHOLD} --- минимальный процент дипломов, необходимый для выделения главного профиля.
    \item \texttt{MIDDLE\_SUBJECT\_THRESHOLD} --- минимальный процент дипломов, необходимый для установки многопрофильности школы.
    \item \texttt{EPS} --- порог разницы процентов между дипломами, необходимый для установки многопрофильности школы.
\end{itemize}

Если значение какой-то координаты в пересчете на проценты превосходит 
\texttt{MAIN\_SUBJECT\_THRESHOLD}, то профиль школы соответствует профилю этой координаты. 

Если значение каких-то двух координат в пересчете на превосходит \texttt{MIDDLE\_SUBJECT\_THRESHOLD}, то школа имеет несколько профилей.

\section{Результаты}
Обработка данных --- Михаил Михайлов\\
Построение классификатора --- Даниил Ельцов\\
Отчет --- Михаил Михайлов

\href{https://github.com/Desiment/ml-study/tree/main/classification-tree}{Код}
\end{document}

